[14:22, 08/02/2022] Mateus Broilo: Teste Ifood:

Primeira Questão: Era sobre interpolação (derivadas e tangentes, basicamente)

#!/bin/python3

import math
import os
import random
import re
import sys


#
# Complete the 'interpolate' function below.
#
# The function is expected to return a STRING.
# The function accepts following parameters:
#  1. INTEGER n
#  2. INTEGER_ARRAY instances
#  3. FLOAT_ARRAY price
#

def interpolate(n, instances, price):
    # Write your code here
    # priceExpected = price[1] + (instances - instances[1])/(instances[2] - instances[1]) * (price[2] - price[1]) -> formula of linear extrapolation
    
    #for i in range(len(instances)):
        #if (price[i] <= 0.0):
            #instances = instances.remove(instances[i])
            #price = price.remove(price[i])
        
    
    if (n > instances[-1]):
        price = round(price[-2] + (n - instances[-2])/(instances[-1] - instances[-2]) * (price[-1] - price[-2]), 2)

    elif (n < instances[0]):
        price = round(price[0] + (n - instances[0])/(instances[1] - instances[0]) * (price[1] - price[0]), 2)
    
    elif (len(instances) == 1):
        price = round(instances, 2)
        
    else:
        
        for i in range(len(instances)):
            if (n == instances[i]):
                price = round(price[i], 2)
                
            elif ((n > instances[i]) and (n < instances[i+1])):
                price = round(price[i] + (n - instances[i])/(instances[i+1] - instances[i]) * (price[i+1] - price[i]), 2)
                

    
    print(price)
    return str(price)
    
if _name_ == '_main_':
    fptr = open(os.environ['OUTPUT_PATH'], 'w')

    n = int(input().strip())

    instances_count = int(input().strip())

    instances = []

    for _ in range(instances_count):
        instances_item = int(input().strip())
        instances.append(instances_item)

    price_count = int(input().strip())

    price = []

    for _ in range(price_count):
        price_item = float(input().strip())
        price.append(price_item)

    result = interpolate(n, instances, price)

    fptr.write(result + '\n')

    fptr.close()

Segunda Questão: Era sobre databases (MySQL) -> Não fiz ou não terminei, mas a idéia geral era:
select 
     customers.name, max(orders.price)
from 
     curstomers
inner join
     orders on (id.orders = id.customers)
group by
     customers.name
where
     diferença de max e min order_data = 10 anos
;

Terceira Questão: Era sobre sobre Joint Probability Distribution
* F(x,y) = C*x^2(1+y), para 0 .LE. x,y .LE. 3, encontrar o valor da constante
* Resposta: A integral disso no intervalo definido tem de ser igual à 1. Calculando, encotra-se C = 2/135.

Quarta Questão: Era sobre theory of moments
* Perguntava qual era o primeiro momento de um distribuição gaussiana do tipo F(x)=exp(-x-A), ou algo assim, e dizia que foram observados os seguintes valores para x: 9, 10 e 11.
* Resposta: O primeiro momento é a média, então (9+10+11)/3 = 10

Quinta Questão: Era sobre distribuição de quantis. Dava uma lista de números e perguntava quais alternativas caiam dentro dos respectivos 1o, 2o e 3o quartís.
* Resposta: Criei um dataframe no pandas com essa lista e usei a função quantile do numpy
* df = pd.DataFrame({'data': lista)
* qtl1 = df.lista.quantile(0.25)
* qtl2 = df.lista.quantile(0.50)
* qtl3 = df.lista.quantile(0.75)
* E compara com as opções

Sexta Questão: Era sobre R quadrado e R quadrado ajustado. Perguntava o que acontece quando se aumenta o número de features em uma regressão linear.

Acho que isso aqui teria ajudado na Primeira Questão:

l1 = [1,2,3,4,5,6]
l2 = [5.25, 0.0, 3.43, 6.78, 0.0, 10.23]

zipped = list(zip(l1,l2))

for row in zipped:
    if row[1] == 0.0:
        zipped.remove(row)
        
zipped